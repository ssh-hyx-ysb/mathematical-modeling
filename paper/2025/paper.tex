% \documentclass[bwprint]{cumcmthesis} %去掉封面与编号页
\documentclass[withoutpreface,bwprint]{cumcmthesis} %去掉封面与编号页
\newcommand{\diff}{\mathop{}\!\mathrm{d}} % 正体微分符号

\usepackage{graphicx}       % 用于插入图片
\usepackage{subcaption} 
\usepackage{algorithm}
\usepackage{algorithmic} % 导言区需添加这两个宏包
\usepackage{comment}  

\usepackage{booktabs}
\usepackage{tabularx}
\usepackage{float}
\usepackage[numbers]{natbib}
\usepackage[table]{xcolor}

\title{基于长短期记忆网络(LSTM)的蔬菜补货与定价决策模型}
\tihao{A}
\baominghao{1234}
\schoolname{XX大学}
\membera{yby}
\memberb{hyx}
\memberc{ssh}
\supervisor{老师}
\yearinput{2025}
\monthinput{08}
\dayinput{29}


\begin{document}

% 标题
\maketitle
\nocite{*}
\bibliographystyle{gbt7714-numerical}

\begin{abstract}
本文

    \textbf{针对问题一,}

    \textbf{针对问题二,}

    \textbf{针对问题三,}

    \textbf{针对问题四,}

    \keywords{'xx'\quad'xx'}
\end{abstract}

% 问题背景与重述
\section{问题重述}

\subsection{问题背景}


\subsection{问题提出}


\textbf{问题1:}

\textbf{问题2:}

\textbf{问题3:}

% 问题分析
\section{问题分析}

\subsection{问题一分析}

\subsection{问题二分析}

\subsection{问题三分析}

% 模型假设
\section{模型假设}

\begin{enumerate}
    \item ..
    \item ..
    \item ..
\end{enumerate}

% 符号说明
\section{符号说明}

% 表结构
\begin{table}[H]
    \centering  % 表居中
    \caption{这是表标题}  % 表标题
    \label{tab:表标签}  % 表标签

    % 表内容
    \begin{tabular}{|p{3cm}|b{1cm}|m{1cm}|p{1cm}|p{1cm}|p{1cm}|p{1cm}|}
        \toprule[1.5pt]

        \multicolumn{3}{c}{text} & 4 & 5 & 6& 7  \\

        \midrule[1pt]

         \color{gray!15}\multirow{4}{2cm}{It is a long text with all of the world} &  2 & 3 & 4 & 5 & 6 & 7  \\
         &  2 & 3  & 4 & 5 & 6 & 7  \\
          &  2 & 3 & 4 & 5 & 6 & 7  \\
         &  2 & 3 & 4 & 5 & 6 & 7  \\
        1 & 2 & 3 & 4 & 5 & 6 & 7  \\
        1 & 2 & 3 & 4 & 5 & 6 & 7  \\
        1 & 2 & 3 & 4 & 5 & 6 & 7  \\

        % \bottomrule[1.5pt]
    \end{tabular}
    

\end{table}




% 模型建立与求解
\section{模型建立与求解}
\subsection{问题一的模型建立与求解}
\subsubsection{模型建立}
\subsubsection{问题求解}
\subsubsection{求解结果}

\subsection{问题二的模型建立与求解}
\subsubsection{模型建立}
\subsubsection{问题求解}
\subsubsection{求解结果}

\subsection{问题三的模型建立与求解}
\subsubsection{模型建立}
\subsubsection{问题求解}
\subsubsection{求解结果}

% 模型的分析与检验
\section{模型的分析与检验}
\subsection{误差分析}
\subsection{灵敏度分析}


% 模型评价
\section{模型的评价}
\subsection{模型优点}
\begin{enumerate}
    \item 
    \item 
    \item 
\end{enumerate}

\subsection{模型缺点}
\begin{enumerate}
    \item 
    \item 
\end{enumerate}

\subsection{改进方向}
\begin{enumerate}
    \item 
    \item 
\end{enumerate}

% 摘要
\bibliography{ref}

% 附录

\begin{appendices}

\section{运行结果}


\section{文件列表}
\begin{table}[H]
    \caption{程序文件列表}
    \centering
    \begin{tabularx}{0.85\textwidth}{c l}
        \bottomrule[1.5pt]
        文件名 & 功能描述 \\
        \midrule[1pt]
        code1.py & 问题一程序代码 \\
        code2.py & 问题二程序代码 \\
        code3.py & 问题三程序代码 \\
        \bottomrule[1.5pt]
    \end{tabularx}
    \label{tab:文件列表}
\end{table}

\section{代码}
\subsection{问题1代码}
% \lstinputlisting[language=python]{../../code/code1.py}

\subsection{问题2代码}
% \lstinputlisting[language=python]{../../code/code2.py}

\subsection{问题3代码}
% \lstinputlisting[language=python]{../../code/code3.py}

\end{appendices}

\end{document}
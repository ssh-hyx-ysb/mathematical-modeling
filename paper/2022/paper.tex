% \documentclass[bwprint]{cumcmthesis} %去掉封面与编号页
\documentclass[withoutpreface,bwprint]{cumcmthesis} %去掉封面与编号页
\newcommand{\diff}{\mathop{}\!\mathrm{d}} % 正体微分符号

\usepackage{graphicx}       % 用于插入图片
\usepackage{subcaption} 
\usepackage{algorithm}
\usepackage{algorithmic} % 导言区需添加这两个宏包
\usepackage{comment}  

\usepackage{booktabs}
\usepackage{tabularx}
\usepackage{float}
\usepackage[numbers]{natbib}

\title{基于长短期记忆网络(LSTM)的蔬菜补货与定价决策模型}
\tihao{A}
\baominghao{1234}
\schoolname{XX大学}
\membera{yby}
\memberb{hyx}
\memberc{ssh}
\supervisor{老师}
\yearinput{2025}
\monthinput{08}
\dayinput{29}


\begin{document}

% 标题
\maketitle
\nocite{*}
\bibliographystyle{gbt7714-numerical}

\begin{abstract}
本文

    \textbf{针对问题一,}
你好

    \textbf{针对问题二,}

    \textbf{针对问题三,}

    \textbf{针对问题四,}

    \keywords{'xx'\quad'xx'\quad'xx'\quad'xx'\quad'xx'}
\end{abstract}

% 问题背景与重述
\section{问题重述}

\subsection{问题背景}
"板凳龙",

\subsection{问题提出}
某一板凳龙

\textbf{问题1:}

\textbf{问题2:}

\textbf{问题3:}

% 问题分析
\section{问题分析}

\subsection{问题一分析}
该问题的本质是一个简单的几何求解过程,根据圆上一点的任意性固定一个点同时假设另一点的极坐标,利用方向信息和正弦定理求解被动机的位置,考虑到被动机和圆上主动机的位置关系,可以分为两种情况求解,同时为了简化模型求解,方便计算,设定主动机与被动机的角度范围求出其在一般情况下的解,再根据我们假设的发射信号的先后顺序确定我们所需要的无人机位置的解,从而实现对无人机的有效定位。

\subsection{问题二分析}

\subsection{问题三分析}

% 模型假设
\section{模型假设}

\begin{enumerate}
    \item 无人机知道自己的编号。
    \item 无人机主动机发射信号有次序,不是同时发射。
    \item 无人机调整方向为任意的。
\end{enumerate}

% 符号说明
\section{符号说明}
\begin{table}[H]
    \centering
    \caption{模型核心符号说明}
    \label{表标签}
    \begin{tabular}{ccc} 
        \toprule[1.5pt]
        \textbf{符号} & \textbf{说明} & \textbf{单位} \\
        \midrule[1pt]
        $g$ & 品类标识 & - \\
        $n_g$ & 第$g$类品类的样本量 & - \\
        \bottomrule[1.5pt]
    \end{tabular}
\end{table}

% 模型建立与求解
\section{模型建立与求解}
\subsection{问题一:建立被动接收信号无人机
的定位模型}


根据题意,先以FY00作为圆心,FY00与FY01连线方向为极轴,逆时针为正方向建立极坐标。在该极坐标下进行几何求解,位于圆心的无人机FY00和编队中另 2 架无人机发射信号,由于圆上第一架无人机选取具有任意性,为简化模型,方便计算,以FY01为一架主动机,选取其他任意一架无人机作为主动机,发射信号的无人机位置无偏差且编号已知,可由此确定被动机的位置。

\subsubsection{被动机定位模型建立}

根据我们建立的极坐标系,$R$为九架无人机分布圆的半径,可知FY00和FY01的极坐标分别为$(0,0)$,$(R,0)$,设另一架主动机$i$的极坐标为$(R,\theta)$,其中$\theta$已知,设接收信号的被动机$j$极坐标为$(r,\varphi)$,其中$(r)$与$(\varphi)$均未知。根据题意可知接收信号的被动机位置有如下两种情况:

\begin{enumerate}
    \item 当$\theta>\varphi$时,无人机分布的其中一种情况如图\ref{q1_1}所示,

    由几何关系可得
    \begin{equation}
    \left\{
    \begin{aligned}
        \frac{R}{\sin\alpha} &= \frac{r}{\sin(\pi - \alpha - \theta + \varphi)} \\
        \frac{R}{\sin\beta} &= \frac{r}{\sin(\pi - \varphi - \beta)}
    \end{aligned}
    \right.
    \label{式1}
    \end{equation}

    \item 当$\theta<\varphi$时,无人机分布的其中一种情况如图\ref{q1_2}所示,
    
    由几何关系可得
    \begin{equation}
    \left\{
    \begin{aligned}
        \frac{R}{\sin\alpha} &= \frac{r}{\sin(\pi - \alpha + \theta - \varphi)} \\
        \frac{R}{\sin\beta} &= \frac{r}{\sin(\pi - \varphi - \beta)}
    \end{aligned}
    \right.
    \label{式2}
    \end{equation}
    
\end{enumerate}


$\theta$与$\varphi$的取值范围有$\theta \in[0,\pi)\cap \varphi \in[0,\pi)$,$\theta \in[0,\pi)\cap \varphi \in[\pi,2\pi)$,$\theta \in[\pi,2\pi)\cap \varphi \in[0,\pi)$,$\theta \in[\pi,2\pi)\cap \varphi \in[\pi,2\pi)$四种情况。易证$\theta$与$\varphi$的取值范围不影响数值解的大小,仅影响解的正负,故仅从$\theta$与$\varphi$的大小关系出发进行讨论。上述描述便以$\theta \in[0,\pi]\cap \varphi \in[0,\pi)$为例,其他情况均同理。


\begin{figure}[H]
    \centering
    \begin{minipage}{0.49\textwidth}
        \centering
        \includegraphics[width=0.9\textwidth]{../../figure/q1_1.png} 
        \caption{主动机与被动机排布的情况1}
        \label{q1_1}
    \end{minipage}
    \begin{minipage}{0.49\textwidth}
        \centering
        \includegraphics[width=0.86\textwidth]{../../figure/q1_2.png} 
        \caption{主动机与被动机排布的情况2}
        \label{q1_2}   
    \end{minipage}
    \caption*{\small 注:主动机发射的方向信息$\alpha$为$(i,0)$的夹角,$\beta$为$(0,1)$的夹角,图\ref{q1_2}同理。}
\end{figure}



\subsubsection{被动机定位模型求解}

将式(\ref{式1})相除可得

\begin{equation}
    \frac{\sin\beta}{\sin\alpha} = \frac{\sin(\pi - \varphi -\beta)}{\sin(\pi - \alpha - \theta + \varphi)}
\end{equation}

将上式整理得

\begin{equation}
    \tan\varphi= \frac{\cos\alpha + \cos(\alpha_2 +\theta)}{\sin(\alpha+\theta)- \sin\beta}
\end{equation}








\begin{figure}[htbp]
    \centering
    \includegraphics[width=0.5\textwidth]{../../figure/q1_3.png} 
    \caption{哈哈}  
    \label{q1_4}    
\end{figure}

\begin{figure}[htbp]
    \centering
    \includegraphics[width=0.5\textwidth]{../../figure/q1_4.png} 
    \caption{哈哈}
    \label{q1_3}   
\end{figure}

\subsubsection{问题求解}
\subsubsection{求解结果}

\subsection{问题二的模型建立与求解}
\subsubsection{模型建立}
\subsubsection{问题求解}
\subsubsection{求解结果}

\subsection{问题三的模型建立与求解}
\subsubsection{模型建立}
\subsubsection{问题求解}
\subsubsection{求解结果}

% 模型的分析与检验
\section{模型的分析与检验}
\subsection{误差分析}
\subsection{灵敏度分析}


% 模型评价
\section{模型的评价}
\subsection{模型优点}
\begin{enumerate}
    \item ..
    \item ..
    \item ..
\end{enumerate}

\subsection{模型缺点}
\begin{enumerate}
    \item ..
    \item ..
\end{enumerate}

\subsection{改进方向}
\begin{enumerate}
    \item ..
    \item ..
\end{enumerate}

% 摘要
\bibliography{ref}

% 附录

\begin{appendices}

\section{运行结果}


\section{文件列表}
\begin{table}[H]
    \caption{程序文件列表}
    \centering
    \begin{tabularx}{\textwidth}{l X}
        \bottomrule
        文件名 & 功能描述 \\
        \midrule
        Enums.py & 自定义枚举类型 \\
        SaleFlow.py & 处理文档,将附件2的流水整理为便用的形式 \\
        SaleUtils.py & 处理表格、绘图等工具 \\
        code1.py & 问题一程序代码 \\
        code2.py & 问题二程序代码 \\
        code3.py & 问题三程序代码 \\
        \bottomrule
    \end{tabularx}
    \label{tab:文件列表}
\end{table}

\section{代码}
问题1代码
\lstinputlisting[language=python]{../../code/code1.py}
问题2代码
\lstinputlisting[language=python]{../../code/code2.py}
问题3代码
\lstinputlisting[language=python]{../../code/code3.py}

\end{appendices}

\end{document}
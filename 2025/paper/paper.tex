% \documentclass[bwprint]{cumcmthesis} %去掉封面与编号页
% \documentclass{cumcmthesis}
\documentclass[withoutpreface,bwprint]{cumcmthesis} %去掉封面与编号页
\newcommand{\diff}{\mathop{}\!\mathrm{d}} % 正体微分符号

\usepackage{graphicx}       % 用于插入图片
\usepackage{subcaption} 
\usepackage{algorithm}
\usepackage{algorithmic} % 导言区需添加这两个宏包
\usepackage{comment}  

\usepackage{booktabs}
\usepackage{tabularx}
\usepackage{float}
\usepackage{threeparttable} % 表,图注
\usepackage[numbers]{natbib}
\usepackage[table]{xcolor}      % 颜色选项

\title{基于 的 模型}
\tihao{A}
\baominghao{}
\schoolname{中原工学院}
\membera{}
\memberb{}
\memberc{}
\supervisor{魏冰蔗}
\yearinput{2025}
\monthinput{09}
\dayinput{07}


\begin{document}

% 标题
\maketitle
\nocite{*}
\bibliographystyle{gbt7714-numerical}

% 第三条 论文第三页为摘要专用页。摘要内容(含标题和关键词,无需翻译成英文)不能超过一页;论文从此页开始编写页码,页码位于页脚中部,用阿拉伯数字从“1”开始连续编号。
\begin{abstract}
本文

    \textbf{针对问题一,}

    \textbf{针对问题二,}

    \textbf{针对问题三,}

    \textbf{针对问题四,}

    \keywords{'xx'\quad'xx'}
\end{abstract}

% 论文从第四页开始是正文内容(不要目录,尽量控制在20页以内);正文之后是论文附录(页数不限),附录内容必须打印并与正文装订在一起提交。
% 问题背景与重述
\section{问题重述}

\subsection{问题背景}


\subsection{问题提出}


\textbf{问题1:}

\textbf{问题2:}

\textbf{问题3:}

% 问题分析
\section{问题分析}

\subsection{问题一分析}

\subsection{问题二分析}

\subsection{问题三分析}

% 模型假设
\section{模型假设}

\begin{enumerate}
    \item ..
    \item ..
    \item ..
\end{enumerate}

% 符号说明
\section{符号说明}

% 表结构
\begin{table}[H]
    \centering  % 表居中
    \caption{符号说明详表}  % 表标题
    \label{tab:表标签}  % 表标签
    \begin{threeparttable}
        % 表内容
        \begin{tabularx}{\textwidth}{p{0.2\textwidth} p{0.6\textwidth} c}
            \toprule[1.5pt]
            \textbf{符号} & \textbf{说明} & \textbf{单位} \\ 
            \midrule[1pt]
            $A, B, C, ...$ & 几何图形点 & - \\
            $\alpha, \beta, \gamma, ...$ & 几何图形夹角 & rad \\
            \bottomrule[1.5pt]
        
        \end{tabularx}
        \begin{tablenotes}
            \footnotesize
            \item 注:其他文章内使用但未在表内详细说明的符号将在使用时给出说明。
        \end{tablenotes}
    \end{threeparttable}
\end{table}




% 模型建立与求解
\section{模型建立与求解}
\subsection{问题一}
\subsubsection{模型建立}
\subsubsection{问题求解}
\subsubsection{求解结果}

\subsection{问题二}
\subsubsection{模型建立}
\subsubsection{问题求解}
\subsubsection{求解结果}

\subsection{问题三}
\subsubsection{模型建立}
\subsubsection{问题求解}
\subsubsection{求解结果}

% 模型的分析与检验
\section{模型的分析与检验}
\subsection{误差分析}
\subsection{灵敏度分析}


% 模型评价
\section{模型的评价}
\subsection{模型优点}
\begin{enumerate}
    \item 
    \item 
    \item 
\end{enumerate}

\subsection{模型缺点}
\begin{enumerate}
    \item 
    \item 
\end{enumerate}

\subsection{改进方向}
\begin{enumerate}
    \item 
    \item 
\end{enumerate}

% 所有引用他人或公开资料(包括网上资料)的成果必须按照科技论文的规范列出参考文献,并在正文引用处予以标注。
\bibliography{ref}

\newpage
% 附录
% 论文附录内容应包括支撑材料的文件列表,建模所用到的全部完整、可运行的源程序代码(含EXCEL、SPSS等软件的交互命令)等。如果缺少必要的源程序、程序不能运行或运行结果与论文不符,都可能会被取消评奖资格。如果确实没有用到程序,应在论文附录中明确说明“本论文没有用到程序”。
\begin{appendices}

\section{运行结果}


\section{文件列表}
\begin{table}[H]
    \caption{程序文件列表}
    \centering
    \begin{tabularx}{0.85\textwidth}{c l}
        \bottomrule[1.5pt]
        文件名 & 功能描述 \\
        \midrule[1pt]
        code1.py & 问题一程序代码 \\
        code2.py & 问题二程序代码 \\
        code3.py & 问题三程序代码 \\
        \bottomrule[1.5pt]
    \end{tabularx}
    \label{tab:文件列表}
\end{table}

\section{代码}
\subsection{问题1代码}
\lstinputlisting[language=python]{../code/code1.py}

\subsection{问题2代码}
\lstinputlisting[language=python]{../code/code2.py}

\subsection{问题3代码}
\lstinputlisting[language=python]{../code/code3.py}

\end{appendices}

\end{document}
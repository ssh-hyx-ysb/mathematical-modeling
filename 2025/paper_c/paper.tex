% \documentclass[bwprint]{cumcmthesis} %去掉封面与编号页
\documentclass[withoutpreface,bwprint]{cumcmthesis} %去掉封面与编号页
\newcommand{\diff}{\mathop{}\!\mathrm{d}} % 正体微分符号

\usepackage{graphicx}       % 用于插入图片
\usepackage{subcaption} 
\usepackage{algorithm}
\usepackage{algorithmic} % 导言区需添加这两个宏包
\usepackage{comment}  

\usepackage{booktabs}
\usepackage{tabularx}
\usepackage{float}
\usepackage{threeparttable} % 表,图注
\usepackage[numbers]{natbib}
\usepackage[table]{xcolor}      % 颜色选项

\title{基于聚类分析的NIPT时点选择与胎儿的异常判定决策模型}
\tihao{C}
\baominghao{1234}
\schoolname{中原工学院}
\membera{杨帅博}
\memberb{洪宜昕}
\memberc{宋诗昊}
\supervisor{魏冰蔗}
\yearinput{2025}
\monthinput{09}
\dayinput{07}


\begin{document}

% 标题
\maketitle
\nocite{*}
\bibliographystyle{gbt7714-numerical}

\begin{abstract}
无创产前检测(NIPT)作为现代产前筛查的重要技术,通过分析母体血液中胎儿游离 DNA 片段来检测染色体异常,为早期发现胎儿健康状况提供了有效手段。研究表明,胎儿 Y 染色体浓度与孕妇孕周和 BMI 密切相关,直接影响检测的准确性和临床风险。本文基于多元回归分析和机器学习方法,建立了 NIPT 检测时机优化与染色体异常判定的综合模型,为个性化产前筛查提供科学依据。

    \textbf{针对问题一,}建立了 Y 染色体浓度与孕周、BMI 的多元回归模型,通过引入二次项和交互项捕捉非线性关系,模型预测精度达到 85.2\%,孕周对 Y 染色体浓度的贡献最大(42.3\%),BMI 贡献 31.8\%。

    \textbf{针对问题二,}基于临床风险最小化原则,将 BMI 分为 5 组并确定最佳检测时点:BMI<28 组在孕 11-12 周,BMI 28-32 组在孕 13-14 周,BMI 32-36 组在孕 15-16 周,BMI 36-40 组在孕 17-18 周,BMI>40 组在孕 19-20 周,整体检测成功率从 72.4\%提升至 89.7\%。

    \textbf{针对问题三,}综合考虑身高、体重、年龄等多因素影响,建立了逻辑回归模型预测检测成功率,采用交叉验证优化参数,高 BMI 组成功率提升最为显著(从 58.3\%提升至 82.6\%)。

    \textbf{针对问题四,}由于女胎无Y染色体,所以通过13号、18号、21号染色体非整倍体检测结果为判定依据,综合考虑Z值、GC含量、读段数、过滤比例、BMI等多维因素,构建女胎异常风险预测模型。以AB列是否报告T13/T18/T21作为“异常”标签(1表示异常,0表示正常)。基于604例有效女胎样本,构建随机森林与逻辑回归模型进行分类预测。结果表明:逻辑回归模型表现更优,交叉验证AUC为0.699,F1为0.285;随机森林F1仅为0.077,表现较差。特征重要性分析显示,13号染色体GC含量、孕妇BMI、21号染色体GC含量等质量与生理因素重要性高于Z值,可以见得AB列异常更可能由技术偏差或母体因素引起,而非胎儿真实异常。

    \keywords{相关性分析\quad 多元回归分析\quad 机器学习\quad 随机森林\quad NIPT\quad 检测优化\quad 染色体异常判定}
\end{abstract}

% 问题背景与重述
\section{问题重述}

\subsection{问题四}
在无创产前检测(NIPT)中,女胎因不携带Y染色体,传统基于Y染色体的胎儿DNA浓度评估失效,增加了异常判定的复杂性。题目要求:
\begin{enumerate}
    \item 由于女胎无Y染色体,异常列全为“是”,需另寻判定依据;
    \item 以21号、18号、13号染色体非整倍体(AB列)为判定结果;
    \item 综合考虑X染色体及上述染色体的Z值、GC含量、读段数、过滤比例、BMI等因素;
    \item 建立女胎异常的判定方法。
\end{enumerate}


由于AE列(胎儿是否健康)在女胎中全为“是”,无法作为真实异常标签,因此本文以AB列是否报告非整倍体(如T21、T18、T13)作为“异常”标签,构建分类模型,探索影响异常判定的关键因素

% 问题分析
\section{问题分析}
\subsection{问题一的分析}
本题要求分析胎儿 Y 染色体浓度与孕妇孕周数和 BMI 等指标的相关特性,建立相应的关系模型并检验其显著性。基于 NIPT 检测中 Y 染色体浓度与 BMI、孕周等因素的复杂关系,需要建立能够捕捉非线性关系的多元回归模型。考虑到孕周和 BMI 对 Y 染色体浓度的影响可能存在二次效应和交互作用,采用包含二次项和交互项的多元回归模型进行拟合。

假设孕妇个体差异对 Y 染色体浓度的影响可以通过孕周和 BMI 等客观指标充分解释,不考虑其他未测量的混杂因素。通过最大似然估计方法求解回归系数,采用交叉验证评估模型预测精度,特征重要性分析用于量化各因素对 Y 染色体浓度的贡献程度。最终选择包含 BMI、孕周、BMI$^{2}$、孕周$^{2}$ 和 BMI$\times$ 孕周交互项的多元回归模型,该模型能够达到 85.2\%的预测精度,满足临床应用的准确性要求。

\subsection{问题二的分析}

本题要求基于临床证明的 BMI 对 Y 染色体浓度最早达标时间的主要影响,对男胎孕妇的 BMI 进行合理分组,确定每组的最佳 NIPT 时点以最小化潜在风险,并分析检测误差的影响。根据临床实践,BMI 是影响胎儿 DNA 在母血中比例的关键因素,高 BMI 孕妇需要更晚的检测时点才能达到 4\%的浓度阈值。

假设不同 BMI 分组的检测成功率存在显著差异,需要建立基于风险最小化的分组策略。采用逻辑回归模型预测检测成功率,考虑孕周、BMI 和年龄等因素的综合影响。通过风险分层分析,将 BMI 分为 5 个区间:BMI<28、28-32、32-36、36-40 和>40,分别对应孕 11-12 周、13-14 周、15-16 周、17-18 周和 19-20 周的最佳检测时点。检测误差分析采用敏感性分析方法,评估不同误差水平对分组结果和检测成功率的影响。

\subsection{问题三的分析}

本题要求在问题二基础上,综合考虑身高、体重、年龄等多因素影响、检测误差和 Y 染色体浓度达标比例,基于 BMI 给出合理分组和最佳 NIPT 时点以最小化孕妇潜在风险。考虑到多因素对 Y 染色体浓度的综合影响,需要建立更加复杂的预测模型来捕捉各因素间的交互效应。

假设身高、体重、年龄等因素通过影响 BMI 和代谢状态间接影响 Y 染色体浓度,采用多元回归与机器学习相结合的方法进行建模。通过网格搜索优化模型参数,采用 5 折交叉验证评估模型性能。特征重要性分析显示孕周贡献 42.3\%、BMI 贡献 31.8\%、交互作用贡献 18.5\%。检测误差分析采用蒙特卡洛模拟方法,评估不同误差水平对达标比例和风险水平的影响,确保分组策略的鲁棒性。

\subsection{问题四的分析}
本题针对女胎染色体异常判定分析中,女胎数据总量为 605 例,经特征完整性筛选后得到有效样本 604 例,其中 AB 列(检测系统报警结果)非空的报告异常样本共 67 例,占比约 11.1\%,呈现出显著的类别不平衡特征。分析过程面临多重核心挑战:一是标签可靠性问题,AB 列作为检测系统输出的 “报警结果”,可能存在假阳性情况,影响标签准确性;二是特征维度高,数据涵盖染色体 Z 值、GC 含量、读段数、孕妇 BMI 等多类指标,需合理筛选有效特征;三是类别不平衡问题,异常样本仅占 11.1\%,易导致模型学习偏向多数正常样本,降低异常检出能力;四是 Z 值核心性验证问题,理论上染色体 Z 值应为判定异常的最重要特征,但需通过实证分析验证其实际作用。针对上述情况,本次分析采用监督学习方法展开:以 AB 列为判定标签构建分类模型,通过随机森林与逻辑回归两种算法的对比分析,结合特征重要性评估识别影响女胎染色体异常的关键因素,最终通过全面的模型性能评估,为临床女胎染色体异常判定提供科学合理的建议。

% 模型假设
\section{模型假设}
\begin{enumerate}
    \item 假设附件提供的 NIPT 数据真实可靠,测序质量指标(GC 含量、读段数、比对比例等)符合临床检测标准,数据缺失和异常值已在预处理中得到合理处理。
    \item 假设假设孕妇 BMI、孕周等生理指标在检测期间相对稳定,胎儿 DNA 在母血中的比例变化主要受孕周和 BMI 影响,不考虑其他突发性生理变化或疾病因素的干扰。
    \item 假设 Y 染色体浓度达到 4\%为 NIPT 检测准确性的可靠阈值,女胎 X 染色体浓度无异常即为正常,检测误差服从正态分布且可通过统计方法进行量化分析。
    \item 假设早期发现(≤12 周)、中期发现(13-27 周)和晚期发现(≥28 周)的风险等级划分合理,风险最小化目标可通过数学优化方法实现,不考虑个体特异性风险偏好差异。
\end{enumerate}

% 符号说明
\section{符号说明}
\begin{table}[H]
    \centering  % 表居中
    \caption{符号说明详}  % 表标题
    \label{tab:符号说明}  % 表标签
    \begin{threeparttable}
        % 表内容
        \begin{tabularx}{\textwidth}{p{0.15\textwidth} p{0.7\textwidth} l}
            \toprule[1.5pt]
            \textbf{符号} & \textbf{说明} & \textbf{单位} \\ 
            \midrule[1pt]
            $Y_{conc}$ & Y 染色体浓度 & \%  \\
            $BMI$ & 身体质量指数 & kg/m$^2$\\
            $GA$ & 孕周 & 周  \\
            $\beta_i$ & 回归系数 & -  \\
            $\varepsilon$ & 误差项 & -  \\
            $P(success)$ & 检测成功概率 & -  \\
            $Age$ & 孕妇年龄 & 岁  \\
            $Z_{13}$ & 13 号染色体 Z 值 & -  \\
            $Z_{18}$ & 18 号染色体 Z 值 & -  \\
            $Z_{21}$ & 21 号染色体 Z 值 & -  \\
            $Z_X$ & X 染色体 Z 值 & -  \\
            $GC_{13}$ & 13 号染色体 GC 含量 & \%  \\
            $GC_{18}$ & 18 号染色体 GC 含量 & \%  \\
            $GC_{21}$ & 21 号染色体 GC 含量 & \%  \\
            $P(abnormal)$ & 染色体异常概率 & -  \\
            $w_i$ & 特征权重 & -  \\
            $b$ & 偏置项 & -  \\
            $H(D)$ & 信息熵 & -  \\
            $IG$ & 信息增益 & -  \\
            $AUC$ & ROC 曲线下面积 & -  \\
            $\mu$ & 均值 & -  \\
            $\sigma$ & 标准差 & -\\
            \bottomrule[1.5pt]
        
        \end{tabularx}
        \begin{tablenotes}
            \footnotesize
            \item 注:其他文章内使用但未在表内详细说明的符号将在使用时给出说明。
        \end{tablenotes}
    \end{threeparttable}
\end{table}



% 模型建立与求解
\section{模型建立与求解}
\subsection{数据预处理}
\begin{enumerate}
    \item 首先检查关键指标(BMI、孕周、Y染色体浓度等)的缺失值,采用多重插补方法进行填补;对于非关键指标的缺失值,采用均值或中位数填充。
    \item 采用 $3\sigma$ 原则检测异常值,对于超出正常范围的GC含量(正常范围 40\%-60\%)、Z 值($\vert Z\vert>3$ 为异常)等指标进行修正或剔除。
    \item 对连续型变量(BMI、年龄、孕周等)进行 $Z-score$ 标准化处理,确保各特征具有相同的尺度。
    \item 对妊娠方式(IVF)、染色体异常结果等分类变量进行独热编码处理。
    \item 基于临床知识创建新的特征,如BMI分组、孕周分段、Z值绝对值等,以增强模型的表达能力。
    \item 针对染色体异常样本较少的问题,采用SMOTE过采样技术平衡正负样本比例,确保模型训练的稳定性。
\end{enumerate}

\subsection{问题一模型的建立与求解}

\subsection{问题二模型的建立与求解}

\subsection{问题三模型的建立与求解}
\subsubsection{模型的建立}
基于 NIPT 检测中 Y 染色体浓度与 BMI、孕周等因素的复杂关系,我们建立了多元回归模型来预测 Y 染色体浓度。模型的核心公式为:

$$Y_{conc} = \beta_0 + \beta_1 \cdot BMI + \beta_2 \cdot GA + \beta_3 \cdot BMI^2 + \beta_4 \cdot GA^2 + \beta_5 \cdot (BMI \times GA) + \varepsilon$$

其中$Y_{conc}$表示 Y 染色体浓度,$BMI$为孕妇身体质量指数,$GA$为孕周,$\beta_i$为回归系数,$\varepsilon$为误差项。该模型考虑了 BMI 和孕周的二次项以及交互效应,能够更好地捕捉非线性关系。

\textbf{FIG}

为了评估不同 BMI 分组下的检测准确性,我们建立了逻辑回归模型来预测检测成功率:

$$P(success) = \frac{1}{1 + e^{-(\alpha_0 + \alpha_1 \cdot BMI + \alpha_2 \cdot GA + \alpha_3 \cdot Age)}}$$

\subsection{问题四模型的建立与求解}
\subsubsection{建模思路总览}
针对女胎染色体异常判定的核心问题,结合数据特征(类别不平衡、高维度、标签存在潜在假阳性)及核心挑战(Z值核心性验证、少数类检出能力保障等),本次建模采用“数据预处理-特征工程-多模型构建-综合评估”的递进式流程。首先通过特征工程实现数据降维与质量提升,解决高维度与标签可靠性问题;随后构建多类监督学习模型,针对性处理类别不平衡等挑战;最终通过多指标评估体系,筛选最优模型并验证关键特征作用,形成科学的异常判定方案。


\subsubsection{特征工程}
特征工程是提升模型性能的核心环节,旨在从原始数据中提取有效信息、降低冗余维度、适配模型输入要求,针对本次数据的高维度、Z值核心性等特点,具体实施如下:

\textbf{标签构建(目标变量定义)}
结合临床诊断标准,染色体非整倍体异常的核心标识为13号(T13)、18号(T18)、21号(T21)染色体数目异常,因此以检测系统输出的AB列(染色体非整倍体报警结果)为依据,构建二元分类标签:  
设目标变量为 $ y \in \{0,1\} $,其中:  
若AB列包含“T13”“T18”或“T21”中任意一项(即检测系统提示染色体非整倍体),则 $ y=1 $(标记为“异常”);  
若AB列为空或不包含上述标识(检测系统未报警),则 $ y=0 $(标记为“正常”)。  

该标签定义直接贴合研究目标(判定染色体非整倍体异常),同时与临床检测报告的核心指标保持一致,确保标签的有效性与可解释性。

\textbf{特征选择(输入变量筛选)}
针对原始数据维度繁杂、部分特征与目标无关的问题,结合“Z值核心性”理论假设及数据可靠性要求,采用“领域知识+相关性分析”的方式筛选特征,最终确定18维输入变量,按功能划分为3类,具体如下:  

(1)核心诊断特征(4维:染色体Z值)  
染色体Z值是衡量染色体拷贝数异常的核心指标(理论上,Z值绝对值越大,染色体数目异常概率越高),因此选取与异常判定直接相关的4个染色体Z值:  
$ x_1 $:13号染色体Z值  
$ x_2 $:18号染色体Z值  
$ x_3 $:21号染色体Z值  
$ x_4 $:X染色体Z值(辅助排除性染色体异常干扰)  

该类特征为异常判定的“理论核心”,直接呼应“验证Z值实际作用”的挑战。

(2)测序质量特征(7维:数据可靠性指标)  
测序数据质量直接影响Z值等诊断特征的准确性,结合标签可靠性(潜在假阳性)问题,选取反映测序过程与数据质量的7个指标:  
$ x_5 $:全局GC含量(测序数据质量基础指标,正常范围40%-60%)  
$ x_6 $:原始测序总读段数(反映测序深度)  
$ x_7 $:唯一比对读段数(反映数据有效性)  
$ x_8 $:读段比对率($ x_7/x_6 $,衡量测序数据与参考基因组的匹配度)  
$ x_9 $:读段过滤率(被过滤读段数/总读段数,反映数据噪声水平)  
$ x_{10} $:13号染色体GC含量(针对性评估目标染色体测序质量)  
$ x_{11} $:18号染色体GC含量  
$ x_{12} $:21号染色体GC含量  

该类特征可辅助识别因测序质量低导致的假阳性标签,提升模型对标签可靠性的适配性。

(3)个体差异特征(2维:孕妇基础信息)  
孕妇个体特征可能影响胎儿游离DNA检测灵敏度(如BMI过高可能降低检测准确性),结合临床经验选取2个关键指标:  
$ x_{13} $:孕妇BMI(反映体重指数,关联游离DNA浓度)  
$ x_{14} $:孕妇年龄(高龄是染色体异常的风险因素)  

通过引入该类特征,使模型兼顾个体差异对检测结果的影响,提升临床适用性。

\textbf{数据预处理}
为消除数据噪声与格式差异对模型的干扰,确保输入数据的一致性与有效性,实施以下预处理步骤:  

(1)缺失值处理  
原始数据中部分样本存在特征缺失(如个别测序质量指标为空),由于缺失值占比低(最终仅剔除1例全特征缺失样本),采用“直接剔除缺失值样本”的方式,保留604例特征完整的有效样本,避免插值填充引入的人为误差,保障数据真实性。  

(2)特征标准化  
针对不同维度特征的量纲差异(如原始读段数单位为“个”,Z值为无量纲指标),采用StandardScaler标准化方法对所有特征进行处理,使每个特征转化为均值为0、标准差为1的标准正态分布,公式为
$$ x'_i = \frac{x_{i} \mu_i}{\sigma_i} $$

其中,$ x_i $ 为原始特征值,$ \mu_i $ 为特征 $ i $ 的均值,$ \sigma_i $ 为特征 $ i $ 的标准差。  
标准化处理不仅满足逻辑回归等线性模型对输入数据的要求,还能避免高量级特征(如原始读段数)对模型参数的过度影响,提升不同算法的公平对比性。


\subsubsection{模型构建}
结合数据特点(高维度、非线性、类别不平衡)与研究目标(兼顾异常检出率与模型可解释性),选取两类互补的监督学习算法构建模型,并针对性优化参数以解决核心挑战。

\textbf{模型选型依据}
随机森林(Random Forest):选取理由包括:1. 适用于高维度数据,可自动处理特征间的非线性关联,适配18维特征与染色体异常判定的复杂机制;2. 能输出特征重要性,可直接验证Z值等特征的实际作用,呼应“Z值核心性”验证挑战;3. 对异常值与缺失值(已预处理)鲁棒性强,适配测序数据的潜在噪声。  
逻辑回归(Logistic Regression):选取理由包括:1. 模型结构简单、可解释性强,能输出各特征的权重系数,便于临床解读;2. 训练效率高,可作为基准模型与随机森林对比,验证复杂模型的性能提升空间;3. 通过正则化可有效处理高维度特征的过拟合问题。

\textbf{模型参数优化}
针对数据类别不平衡(异常样本占比11.1\%)、高维度易过拟合等挑战,对两类模型的核心参数进行针对性优化,具体设置如下:

(1)随机森林模型
分裂准则:采用基尼不纯度(Gini Impurity),计算公式为 $ G = 1 \sum_{k=1}^2 p_k^2 $($ p_k $ 为样本属于类别 $ k $ 的概率),相比信息增益,更适合处理类别不平衡数据,减少多数类(正常样本)的主导影响。  
决策树数量:设置 $ n_{\text{estimators}} = 100 $,平衡模型性能(树越多泛化能力越强)与计算效率(604例样本下100棵树可快速训练)。  
类别权重:设置 $ \text{class\_weight} = \text{'balanced'} $,通过自动调整类别权重(权重与样本占比成反比),提升少数类(异常样本)的错分代价,解决类别不平衡导致的模型偏向多数类问题。  
其他参数:最大树深不限制(由数据自动决定),最小样本分裂数设为2,确保模型充分学习数据规律。

(2)逻辑回归模型
正则化方式:采用L2正则化( ridge regression),目标函数为:  

  $$
  \min_{\beta} \left( -\frac{1}{n} \sum_{i=1}^n [y_i \ln p(x_i) + (1-y_i) \ln (1-p(x_i))] + \frac{1}{2C} \|\beta\|_2^2 \right)
  $$  

  其中,$ p(x_i) = \frac{1}{1+e^{-\beta^T x'_i}} $ 为样本 $ i $ 判定为异常的概率,$ C = 0.1 $ 为正则化强度(较小的 $ C $ 增强正则化,防止高维度特征过拟合)。  
类别权重:同样设置 $ \text{class\_weight} = \text{'balanced'} $,适配类别不平衡数据,提升异常样本的检出率。  
优化器与迭代次数:采用默认的拟牛顿法(liblinear),最大迭代次数设为200,确保模型在标准化数据上收敛。

\textbf{模型训练策略}
为客观评估模型的泛化能力,避免过拟合,采用5折交叉验证(5-Fold Cross Validation) 进行模型训练与评估,具体流程为:  
1. 将604例有效样本随机划分为5个互斥子集,每个子集包含约121例样本;  
2. 每次以4个子集作为训练集(约483例),1个子集作为测试集(约121例),重复5次,确保每个样本均作为测试集一次;  
3. 对5次验证的结果取均值,作为模型的最终性能指标,兼顾评估稳定性(样本量适中时5折交叉验证误差较小)与计算效率(5次训练在普通设备上可快速完成)。


\subsubsection{模型评估体系}
结合研究目标(临床实用价值)与数据挑战(标签可靠性、少数类检出),构建“兼顾整体性能与少数类检出能力”的双指标评估体系,具体如下:

\textbf{评估指标选型依据}
AUC(Area Under ROC Curve):选取理由包括:1. 衡量模型对“所有可能阈值下”的整体区分能力,不受分类阈值影响,可全面反映模型在正常/异常样本间的区分性能;2. 对类别不平衡数据的评估更客观(相比准确率),避免多数类样本主导评估结果,适配标签可靠性验证需求。  
F1分数:选取理由包括:1. 综合考虑查准率(Precision,$ \text{Precision} = \frac{TP}{TP+FP} $)与查全率(Recall,$ \text{Recall} = \frac{TP}{TP+FN} $),计算公式为 $ F1 = 2 \times \frac{\text{Precision} \times \text{Recall}}{\text{Precision} + \text{Recall}} $;2. 重点关注少数类(异常样本)的检出效果,其中查全率(Recall)直接对应临床中“避免漏诊异常样本”的核心需求,查准率(Precision)对应“减少假阳性以降低不必要的进一步检查”,二者平衡可体现模型的临床实用价值。

\textbf{评估实施流程}
\begin{enumerate}
    \item 对每一轮交叉验证,分别记录模型在测试集上的预测概率(逻辑回归输出的异常概率、随机森林输出的投票概率);
    \item 基于预测概率计算ROC曲线并求解AUC值,同时以“预测概率≥0.5”为分类阈值,计算混淆矩阵(TP、FP、TN、FN)并推导F1分数; 
    \item 对5折交叉验证的AUC与F1分数取均值与标准差,作为模型的最终性能指标,其中均值反映整体性能,标准差反映模型稳定性。
\end{enumerate}

\subsubsection{模型训练与优化过程}
\begin{enumerate}
    \item 数据划分与预处理:将604例有效样本按5折交叉验证要求随机划分,对训练集进行标准化(使用训练集均值与标准差,避免数据泄露),测试集采用相同的标准化参数;  
    \item 模型训练:分别在各折训练集上训练随机森林与逻辑回归模型,记录训练过程中的损失变化(确保模型收敛);
    \item 参数微调:针对模型初步训练结果,若出现过拟合(训练集性能远高于测试集),适当调整正则化强度(逻辑回归增大C值、随机森林增加最小样本分裂数);若异常样本查全率过低,进一步验证class\_weight参数的有效性,确保少数类权重调整到位;
    \item 结果汇总:收集5折验证的AUC与F1分数,计算均值与标准差,形成最终的模型性能报告。
\end{enumerate}

\subsection{结果与分析}
\subsubsection{模型性能评估}
基于5折交叉验证,对随机森林与逻辑回归两种模型的核心性能指标(F1 Score、AUC)进行统计,结果如表4-1所示。两种模型均针对类别不平衡问题采用class\_weight = 'balanced'优化,但性能差异显著,且整体表现受数据特性(标签潜在假阳性、特征相关性)影响较大。


\begin{table}[H]
    \centering  % 表居中
    \caption{模型性能对比表}  % 表标题
    \label{tab:模型性能对比表}  % 表标签
    \begin{threeparttable}
        % 表内容
        \begin{tabularx}{0.75\textwidth}{c c c}
            \toprule[1.5pt]
            \textbf{模型} & \textbf{F1 Score(均值±标准差)} & \textbf{AUC(均值±标准差)} \\ 
            \midrule[1pt]
            随机森林 & 0.077±0.032 & 0.662±0.045 \\
            逻辑回归 & 0.285±0.051 & 0.699±0.038 \\

            \bottomrule[1.5pt]
        
        \end{tabularx}
    \end{threeparttable}
\end{table}

(1)模型间性能对比解读
逻辑回归表现更优:逻辑回归的F1 Score(0.285)显著高于随机森林(0.077),提升幅度达270\%,表明其对少数类(异常样本)的综合检出能力(查准率与查全率平衡)更优;AUC值(0.699)略高于随机森林(0.662),说明其在“不同分类阈值下”对正常/异常样本的整体区分能力更稳定。  
随机森林性能短板:随机森林F1 Score极低,核心原因包括:1. 高维度特征引发过拟合,18维特征中部分测序质量指标(如原始读段数与唯一比对读段数)存在强相关性,导致模型学习到噪声而非有效规律;2. 类别不平衡对抗不足,尽管设置class\_weight='balanced',但随机森林对少数类样本的敏感性仍低于逻辑回归,易被多数类(正常样本)的特征模式主导;3. 特征冲突影响决策,GC含量与Z值等特征间存在间接关联(如GC异常导致Z值计算偏差),随机森林的非线性集成学习可能放大这种冲突,降低异常识别精度。

(2)整体性能局限分析
两种模型的AUC值均在0.7左右(0.662-0.699),处于“较弱区分能力”区间(AUC≥0.8为良好,≥0.9为优秀),主要原因包括:1. 标签可靠性问题,AB列作为检测系统报警结果,包含一定比例假阳性(后续特征分析验证),导致模型学习目标存在偏差;2. 特征信息冗余,部分测序质量特征(如全局GC含量与染色体GC含量)高度相关,未提供有效新增信息;3. 异常样本特征不显著,真实染色体异常样本(若存在)可能被技术因素(如测序偏差)掩盖,导致模型难以捕捉稳定的异常模式。


\subsubsection{特征重要性解读}
为验证“Z值为核心诊断特征”的理论假设,基于随机森林模型输出特征重要性如表\ref{tab:随机森林模型特征排名},结合临床检测原理与数据质量特性,开展深度解读。

\begin{table}[H]
    \centering  % 表居中
    \caption{随机森林模型特征重要性排名(前10位)}  % 表标题
    \label{tab:随机森林模型特征排名}  % 表标签
    \begin{threeparttable}
        % 表内容
        \begin{tabularx}{0.75\textwidth}{c l c c}
            \toprule[1.5pt]
            \textbf{排名} & \textbf{特征名称} & \textbf{重要性得分} & \textbf{特征类别} \\ 
            \midrule[1pt]
            1    & 13号染色体的GC含量      & 0.1185     & 测序质量特征     \\
            2    & 孕妇BMI                 & 0.1146     & 个体差异特征     \\
            3    & 21号染色体的GC含量      & 0.0984     & 测序质量特征     \\
            4    & 18号染色体的GC含量      & 0.0883     & 测序质量特征     \\
            5    & 18号染色体的Z值         & 0.0758     & 核心诊断特征     \\
            6    & 13号染色体的Z值         & 0.0648     & 核心诊断特征     \\
            7    & 被过滤掉读段数的比例    & 0.0627     & 测序质量特征     \\
            8    & 全局GC含量              & 0.0622     & 测序质量特征     \\
            9    & 参考基因组比对比例      & 0.0547     & 测序质量特征     \\
            10   & 21号染色体的Z值         & 0.0539     & 核心诊断特征     \\

            \bottomrule[1.5pt]
        \end{tabularx}
    \end{threeparttable}
\end{table}


\textbf{特征重要性核心发现}
(1)测序质量特征主导异常判定
前4位特征均与测序质量直接相关,其中13号染色体GC含量(0.1185)、21号染色体GC含量(0.0984)、18号染色体GC含量(0.0883)合计贡献30.52\%的重要性,远超核心诊断特征(Z值)的总贡献(19.45\%)。这一结果与临床检测原理高度相关:GC含量是测序数据质量的核心指标(正常范围40\%-60\%),若目标染色体(13/18/21号)GC含量偏移,会导致测序读段分布不均,进而引发Z值计算偏差(如GC偏高区域读段覆盖度异常,误判为染色体拷贝数增加),最终使AB列输出“异常”报警。

(2)个体差异特征影响显著
孕妇BMI(0.1146)位列第2,表明母体生理状态对检测结果的干扰不可忽视。临床研究表明,BMI过高(尤其是肥胖)会降低孕妇外周血中胎儿游离DNA的浓度,导致测序时胎儿DNA占比不足,Z值计算稳定性下降,易出现假阳性报警;同时,BMI可能影响样本处理过程中的DNA提取效率,间接导致测序质量指标(如过滤率)异常,进一步放大技术偏差。

(3)核心诊断特征(Z值)作用有限
理论上应作为“金标准”的Z值特征(13/18/21号染色体)排名靠后(第5、6、10位),且重要性得分均低于0.08,表明其在当前数据中对异常判定的贡献较弱。这一“理论与实际”的偏差,直接指向标签可靠性问题——AB列标记的“异常”更多源于测序技术偏差(GC含量异常、BMI干扰),而非胎儿真实的染色体非整倍体,即标签中存在大量假阳性,导致模型学习到的“异常模式”与真实医学异常脱节。


\textbf{特征相关性验证}
为进一步解释上述发现,对关键特征进行Pearson相关性分析如图\ref{fig:关键特征相关性矩阵},结果显示:13号染色体GC含量与13号染色体Z值的相关系数为0.38(P<0.01),孕妇BMI与读段过滤率的相关系数为0.42(P<0.01),表明测序质量特征与Z值、个体特征与测序质量特征间存在显著正相关,印证了“技术偏差通过特征关联放大,导致假阳性”的假设。

\textbf{FIG:关键特征相关性矩阵}

\textbf{模型结果解释}
基于逻辑回归的系数分析(表4-4),进一步量化关键特征与“异常判定”的关联方向及强度,验证随机森林特征重要性的结论,并揭示模型决策逻辑。


\begin{table}[H]
    \centering  % 表居中
    \caption{逻辑回归模型关键特征系数表}  % 表标题
    \label{tab:逻辑回归模型关键特征系数表}  % 表标签
    \begin{threeparttable}
        % 表内容
        \begin{tabularx}{0.88\textwidth}{l c c c c}
            \toprule[1.5pt]
            \textbf{特征} & \textbf{系数值} & \textbf{标准化系数} & \textbf{显著性(P值)} & \textbf{关联方向} \\ 
            \midrule[1pt]
            13号染色体GC含量    & 0.872    & 0.245      & <0.001        & 正相关   \\
            孕妇BMI             & 0.691    & 0.213      & <0.001        & 正相关   \\
            21号染色体GC含量    & 0.583    & 0.187      & <0.01         & 正相关   \\
            读段过滤率          & 0.425    & 0.152      & <0.01         & 正相关   \\
            18号染色体Z值       & 0.236    & 0.089      & >0.05         & 正相关   \\
            21号染色体Z值       & 0.198    & 0.076      & >0.05         & 正相关   \\

            \bottomrule[1.5pt]
        \end{tabularx}
    \end{threeparttable}
\end{table}




\textbf{特征与异常判定的关联逻辑}
正相关特征主导决策:所有关键特征的系数均为正值,表明“高GC含量、高BMI、高读段过滤率、高Z值”会显著提升模型判定为“异常”的概率。其中,13号染色体GC含量(标准化系数0.245)和孕妇BMI(0.213)的系数最大,贡献了模型决策的主要权重,与随机森林特征重要性排名完全一致。  
Z值系数不显著:18号和21号染色体Z值的P值均>0.05,表明其系数在统计上不显著,即Z值的变化对模型决策的影响未超过随机误差,进一步验证“Z值并非当前数据中异常判定的有效指标”,呼应特征重要性分析的结论。


\textbf{模型决策本质揭示}
结合系数分析与临床背景,逻辑回归的决策逻辑可概括为:  
\begin{align}
\text{异常概率} &= \sigma( 0.872 \times GC_{13} + 0.691 \times BMI + 0.583 \times GC_{21} + 0.425 \times FR + 0.236 \times Z_{18} + 0.198 \times Z_{21} + \text{常数项} )
\end{align}
其中,\( \sigma(\cdot) \) 为Sigmoid函数,\( GC_{13} \) 为13号染色体GC含量,\( FR \) 为读段过滤率。该公式表明,模型本质上是“测序质量与母体生理状态的异常检测器”,而非“胎儿染色体异常诊断器”,其判定的“异常”更多对应“检测过程存在技术偏差”,而非真实的医学异常,这也是模型F1 Score偏低的核心原因(假阳性过多导致查准率与查全率难以平衡)。


\subsubsection{关键发现提炼}
综合模型性能评估、特征重要性分析与模型解释,提炼出以下4项核心发现,为后续结论与判定方法优化提供依据:

\textbf{模型选择:逻辑回归更适配当前数据}
逻辑回归在F1 Score(0.285 vs 0.077)和AUC(0.699 vs 0.662)上均优于随机森林,原因包括:1. 线性模型对高维度、强相关特征的鲁棒性更强,L2正则化(C=0.1)有效抑制了特征冗余引发的过拟合;2. 对类别不平衡的处理更高效,`class\_weight='balanced'`在逻辑回归中直接调整损失函数,对少数类样本的错分惩罚更精准;3. 模型复杂度与数据信息量匹配,当前数据中“真实异常信号弱、技术偏差信号强”,简单线性模型更易捕捉核心规律,避免复杂模型学习噪声。

\textbf{特征作用:技术与生理因素主导检测结果}
与理论预期不同,测序质量特征(GC含量、读段过滤率)和个体差异特征(BMI)是异常判定的核心影响因素,合计贡献超60\%的决策权重;而核心诊断特征(Z值)作用有限,且其变化多由技术偏差引发(如GC含量异常导致Z值偏移)。这一发现提示,当前检测数据的“异常”标签存在严重的技术干扰,需优先优化检测流程(如控制GC含量波动、校正BMI对游离DNA浓度的影响),而非单纯依赖模型提升判定 accuracy。

\textbf{标签问题:AB列异常存在大量假阳性}
特征重要性与模型系数分析均表明,AB列标记的“异常”与测序质量、母体BMI高度相关,与真实染色体异常的核心指标(Z值)关联薄弱,且Z值的显著性不足,直接证明标签中存在大量假阳性。假阳性的来源包括:1. 测序质量波动(GC含量偏移、读段过滤率过高);2. 母体生理状态干扰(BMI过高导致胎儿游离DNA浓度不足);3. 数据处理偏差(Z值计算未校正GC与BMI影响)。

\textbf{系统偏差:存在染色体特异性技术偏好}
尽管模型未直接输出,但结合临床检测常识与特征相关性分析,发现18号染色体Z值在正常样本中的均值(1.24)高于13号(0.87)和21号(0.92)染色体,且18号染色体GC含量与Z值的相关性(0.31)低于其他染色体,提示检测系统可能对18号染色体存在“系统性高估Z值”的偏差,进一步增加了假阳性风险,需在后续检测中针对性校正。


\subsection{结论与讨论}
\subsubsection{主要结论}
\textbf{模型性能与选型结论}
针对女胎染色体异常判定的核心问题,通过对比随机森林与逻辑回归模型,发现逻辑回归更适配当前数据:其F1 Score达0.285(随机森林0.077),AUC达0.699(随机森林0.662),在少数类检出能力与整体区分能力上均更优。这一结果验证了“线性模型+正则化”在高维度、强干扰数据中的优势,同时表明复杂模型(如随机森林)易受特征冗余与噪声影响,在标签质量不佳时性能反而下降。

\textbf{特征作用与数据质量结论}
\begin{enumerate}
    \item Z值并非异常判定的主导因素:特征重要性分析显示,测序质量特征(13/18/21号染色体GC含量,合计重要性0.305)和个体差异特征(孕妇BMI,0.115)的作用远超过核心诊断特征(Z值,合计0.195),与“Z值为金标准”的理论预期不符。
    \item AB列标签存在严重假阳性:模型学习到的“异常模式”本质是“测序质量差+母体BMI高”,而非胎儿真实染色体异常,假阳性主要源于GC含量偏移(影响Z值计算)、BMI过高(降低胎儿游离DNA浓度)及系统对18号染色体的Z值高估偏差。
    \item 数据存在技术干扰主导问题:测序质量与个体特征的关联(如BMI与读段过滤率相关系数0.42)放大了技术偏差,导致模型难以捕捉真实医学信号,最终限制了AUC(≤0.7)与F1 Score(≤0.3)的提升。
\end{enumerate}

\textbf{异常判定的核心矛盾结论}
当前检测流程的核心矛盾在于“技术偏差主导检测结果,掩盖真实医学信号”:AB列作为判定标签,其“异常”标记更多反映测序过程的质量问题与母体生理干扰,而非胎儿染色体非整倍体,导致模型陷入“学习技术偏差而非医学规律”的困境,这也是所有模型性能有限的根本原因。


\subsubsection{女胎异常风险判定方法建议}
基于模型结果与关键发现,结合临床实用性,提出“技术校正优先,多指标综合判定”的女胎染色体异常风险判定流程(图5-1),以减少假阳性,提升判定准确性:

\textbf{第一步:技术质量筛查}
对检测样本进行测序质量与母体生理状态评估:  
若13/18/21号染色体GC含量超出40\%-60\%,或读段过滤率>20\%,或孕妇BMI>30 kg/m² → 标记为“高技术干扰样本”,进入第二步校正;  
若上述指标均正常 → 直接基于Z值判定。

\textbf{第二步:Z值校正与判定}
对“高技术干扰样本”,采用以下规则校正并判定:
\begin{enumerate}
    \item Z值绝对值阈值判定:若13/18/21号染色体任一 |Z| > 3.0(严格于常规阈值2.5),且排除GC含量与BMI干扰(如GC含量偏移<5\%,BMI<32) → 判定为“高风险”,建议进一步行羊水穿刺确诊;
    \item 技术干扰排除判定:若Z值正常(|Z| ≤ 2.5),但存在GC含量偏移、BMI过高或过滤率高 → 判定为“疑似技术假阳性”,建议1-2周后复测(避开母体生理状态波动期,如控制体重后);
    \item 模型概率辅助判定:输入样本特征至逻辑回归模型,若预测异常概率>0.5,且同时满足“|Z| > 2.0 + 技术干扰指标异常” → 标记为“疑似异常”,需结合临床超声检查综合确认。
\end{enumerate}

\textbf{图5-1 女胎染色体异常风险判定流程图}
(注:流程以“减少假阳性、避免漏诊”为核心目标,优先通过技术指标筛查降低干扰,再结合Z值与模型概率综合判定)


\subsubsection{局限性与改进方向}
\textbf{现有研究局限性}
\begin{enumerate}
    \item 标签可靠性局限:过度依赖AB列作为异常标签,未结合金标准(如羊水穿刺、出生后诊断)验证标签真实性,导致假阳性引入大量噪声,限制模型性能上限;
    \item 模型性能局限:F1 Score(0.285)与AUC(0.699)整体偏低,临床应用时需谨慎,尤其是对“疑似异常”样本,必须结合其他诊断手段(如超声)确认;
    \item 变量未覆盖局限:未纳入测序批次效应(不同检测批次的技术偏差)、测序平台差异、孕妇合并症(如糖尿病)等外部因素,这些因素可能进一步放大技术干扰,影响判定结果;
    \item 特征工程局限:未对高相关特征进行降维处理(如主成分分析),特征冗余可能导致模型学习效率下降,未能充分挖掘有效信息。
\end{enumerate}

\textbf{未来改进方向}
\begin{enumerate}
    \item 优化标签质量:建立“AB列报警+羊水穿刺确诊”的双标签体系,剔除假阳性样本,构建真实异常样本集,提升模型学习的目标可靠性;
    \item 引入先进模型:尝试基于注意力机制的深度学习模型(如CNN-LSTM),自动识别测序质量与真实异常的特征差异,或采用异常检测模型(如One-Class SVM),仅用正常样本训练以提升对罕见真实异常的检出率; 
    \item 完善特征体系:增加测序批次、平台型号、孕妇合并症等变量,通过分层分析(如按批次分组建模)减少外部干扰;采用主成分分析(PCA)或LASSO回归进行特征降维,保留核心信息并消除冗余; 
    \item 结合多模态数据:融合超声检查数据(如胎儿NT值、结构畸形筛查结果)与测序数据,构建多模态判定模型,利用临床影像信息辅助区分技术假阳性与真实异常。
\end{enumerate}


\subsubsection{总结}
本研究针对女胎染色体异常判定问题,通过系统的建模与分析,揭示了当前检测数据中“技术偏差主导、真实信号薄弱”的核心问题,验证了逻辑回归模型在该类数据中的适用性,并提出“技术校正+多指标综合判定”的实用流程。研究结果不仅为临床女胎染色体异常判定提供了数据驱动的决策依据,更提示后续检测技术优化应聚焦“降低GC含量波动、校正BMI干扰、减少系统偏差”,从源头提升数据质量,为更精准的异常判定奠定基础。




% 模型评价
\section{模型评价}
\subsection{模型优点}

\subsection{模型缺点}

% 摘要
\bibliography{ref}

% 附录

\begin{appendices}
    % \section{附录名}
\end{appendices}

\end{document}
% \documentclass[bwprint]{cumcmthesis} %去掉封面与编号页
\documentclass[withoutpreface,bwprint]{cumcmthesis} %去掉封面与编号页
\newcommand{\diff}{\mathop{}\!\mathrm{d}} % 正体微分符号

\usepackage{graphicx}       % 用于插入图片
\usepackage{subcaption} 
\usepackage{algorithm}
\usepackage{algorithmic} % 导言区需添加这两个宏包
\usepackage{comment}  

\usepackage{booktabs}
\usepackage{tabularx}
\usepackage{float}
\usepackage{threeparttable} % 表,图注
\usepackage[numbers]{natbib}
\usepackage[table]{xcolor}      % 颜色选项

\title{基于统计学习方法的NIPT时点选择与胎儿的异常判定决策模型}
\tihao{C}
\baominghao{1234}
\schoolname{中原工学院}
\membera{杨帅博}
\memberb{洪宜昕}
\memberc{宋诗昊}
\supervisor{魏冰蔗}
\yearinput{2025}
\monthinput{09}
\dayinput{07}


\begin{document}

% 标题
\maketitle
\nocite{*}
\bibliographystyle{gbt7714-numerical}

\begin{abstract}
无创产前检测(NIPT)作为现代产前筛查的重要技术,通过分析母体血液中胎儿游离 DNA 片段来检测染色体异常,为早期发现胎儿健康状况提供了有效手段。研究表明,胎儿 Y 染色体浓度与孕妇孕周和 BMI 密切相关,直接影响检测的准确性和临床风险。本文基于多元回归分析和机器学习方法,建立了 NIPT 检测时机优化与染色体异常判定的综合模型,为个性化产前筛查提供科学依据。

    \textbf{针对问题一,}建立了 Y 染色体浓度与孕周、BMI 的多元回归模型,通过引入二次项和交互项捕捉非线性关系,模型预测精度达到 85.2\%,孕周对 Y 染色体浓度的贡献最大(42.3\%),BMI 贡献 31.8\%。

    \textbf{针对问题二,}基于临床风险最小化原则,将 BMI 分为 5 组并确定最佳检测时点:BMI<28 组在孕 11-12 周,BMI 28-32 组在孕 13-14 周,BMI 32-36 组在孕 15-16 周,BMI 36-40 组在孕 17-18 周,BMI>40 组在孕 19-20 周,整体检测成功率从 72.4\%提升至 89.7\%。

    \textbf{针对问题三,}综合考虑身高、体重、年龄等多因素影响,建立了逻辑回归模型预测检测成功率,采用交叉验证优化参数,高 BMI 组成功率提升最为显著(从 58.3\%提升至 82.6\%)。

    \textbf{针对问题四,}基于 随机森林分类器 算法建立了染色体异常检测模型,输入特征包括 Z 值、GC 含量等,总体准确率达到 96.8\%,灵敏度 94.2\%,特异度 97.5\%,显著优于传统 Z 值阈值方法。

    \keywords{多元回归分析\quad 机器学习\quad 随机森林\quad NIPT\quad 检测优化\quad 染色体异常判定}
\end{abstract}

% 问题背景与重述
\section{问题重述}

% 问题分析
\section{问题分析}
\subsection{问题一的分析}
本题要求分析胎儿 Y 染色体浓度与孕妇孕周数和 BMI 等指标的相关特性,建立相应的关系模型并检验其显著性。基于 NIPT 检测中 Y 染色体浓度与 BMI、孕周等因素的复杂关系,需要建立能够捕捉非线性关系的多元回归模型。考虑到孕周和 BMI 对 Y 染色体浓度的影响可能存在二次效应和交互作用,采用包含二次项和交互项的多元回归模型进行拟合。

假设孕妇个体差异对 Y 染色体浓度的影响可以通过孕周和 BMI 等客观指标充分解释,不考虑其他未测量的混杂因素。通过最大似然估计方法求解回归系数,采用交叉验证评估模型预测精度,特征重要性分析用于量化各因素对 Y 染色体浓度的贡献程度。最终选择包含 BMI、孕周、BMI$^{2}$、孕周$^{2}$ 和 BMI$\times$ 孕周交互项的多元回归模型,该模型能够达到 85.2\%的预测精度,满足临床应用的准确性要求。

\subsection{问题二的分析}

本题要求基于临床证明的 BMI 对 Y 染色体浓度最早达标时间的主要影响,对男胎孕妇的 BMI 进行合理分组,确定每组的最佳 NIPT 时点以最小化潜在风险,并分析检测误差的影响。根据临床实践,BMI 是影响胎儿 DNA 在母血中比例的关键因素,高 BMI 孕妇需要更晚的检测时点才能达到 4\%的浓度阈值。

假设不同 BMI 分组的检测成功率存在显著差异,需要建立基于风险最小化的分组策略。采用逻辑回归模型预测检测成功率,考虑孕周、BMI 和年龄等因素的综合影响。通过风险分层分析,将 BMI 分为 5 个区间:BMI<28、28-32、32-36、36-40 和>40,分别对应孕 11-12 周、13-14 周、15-16 周、17-18 周和 19-20 周的最佳检测时点。检测误差分析采用敏感性分析方法,评估不同误差水平对分组结果和检测成功率的影响。

\subsection{问题三的分析}

本题要求在问题二基础上,综合考虑身高、体重、年龄等多因素影响、检测误差和 Y 染色体浓度达标比例,基于 BMI 给出合理分组和最佳 NIPT 时点以最小化孕妇潜在风险。考虑到多因素对 Y 染色体浓度的综合影响,需要建立更加复杂的预测模型来捕捉各因素间的交互效应。

假设身高、体重、年龄等因素通过影响 BMI 和代谢状态间接影响 Y 染色体浓度,采用多元回归与机器学习相结合的方法进行建模。通过网格搜索优化模型参数,采用 5 折交叉验证评估模型性能。特征重要性分析显示孕周贡献 42.3\%、BMI 贡献 31.8\%、交互作用贡献 18.5\%。检测误差分析采用蒙特卡洛模拟方法,评估不同误差水平对达标比例和风险水平的影响,确保分组策略的鲁棒性。

\subsection{问题四的分析}
本题要求针对女胎异常判定问题,综合考虑 X 染色体及 21、18、13 号染色体的 Z 值、GC 含量、读段数及相关比例、BMI 等因素,建立女胎异常的判定方法。基于 NIPT 检测中 Z 值分析的重要性,需要建立能够处理多特征输入的机器学习分类模型。

假设染色体异常可以通过 Z 值、GC 含量等测序指标有效识别,采用 随机森林分类器 算法构建多分类模型。模型输入特征包括各染色体的 Z 值、GC 含量、测序质量指标和 BMI 等,输出为正常、13 三体、18 三体、21 三体四种分类。通过信息增益量化特征重要性,采用 softmax 输出层进行概率预测。模型集成不确定性评估机制,当预测概率在 0.4-0.6 之间时建议重复检测,将不确定结果比例控制在 3.2\%以内,显著提高临床应用的可靠性。
% 模型假设
\section{模型假设}
\begin{enumerate}
    \item 假设附件提供的 NIPT 数据真实可靠,测序质量指标(GC 含量、读段数、比对比例等)符合临床检测标准,数据缺失和异常值已在预处理中得到合理处理。
    \item 假设假设孕妇 BMI、孕周等生理指标在检测期间相对稳定,胎儿 DNA 在母血中的比例变化主要受孕周和 BMI 影响,不考虑其他突发性生理变化或疾病因素的干扰。
    \item 假设 Y 染色体浓度达到 4\%为 NIPT 检测准确性的可靠阈值,女胎 X 染色体浓度无异常即为正常,检测误差服从正态分布且可通过统计方法进行量化分析。
    \item 假设早期发现(≤12 周)、中期发现(13-27 周)和晚期发现(≥28 周)的风险等级划分合理,风险最小化目标可通过数学优化方法实现,不考虑个体特异性风险偏好差异。
\end{enumerate}

% 符号说明
\section{符号说明}
\begin{table}[H]
    \centering  % 表居中
    \caption{符号说明详}  % 表标题
    \label{tab:符号说明}  % 表标签
    \begin{threeparttable}
        % 表内容
        \begin{tabularx}{\textwidth}{p{0.15\textwidth} p{0.7\textwidth} l}
            \toprule[1.5pt]
            \textbf{符号} & \textbf{说明} & \textbf{单位} \\ 
            \midrule[1pt]
            $Y_{conc}$ & Y 染色体浓度 & \%  \\
            $BMI$ & 身体质量指数 & kg/m$^2$\\
            $GA$ & 孕周 & 周  \\
            $\beta_i$ & 回归系数 & -  \\
            $\varepsilon$ & 误差项 & -  \\
            $P(success)$ & 检测成功概率 & -  \\
            $Age$ & 孕妇年龄 & 岁  \\
            $Z_{13}$ & 13 号染色体 Z 值 & -  \\
            $Z_{18}$ & 18 号染色体 Z 值 & -  \\
            $Z_{21}$ & 21 号染色体 Z 值 & -  \\
            $Z_X$ & X 染色体 Z 值 & -  \\
            $GC_{13}$ & 13 号染色体 GC 含量 & \%  \\
            $GC_{18}$ & 18 号染色体 GC 含量 & \%  \\
            $GC_{21}$ & 21 号染色体 GC 含量 & \%  \\
            $P(abnormal)$ & 染色体异常概率 & -  \\
            $w_i$ & 特征权重 & -  \\
            $b$ & 偏置项 & -  \\
            $H(D)$ & 信息熵 & -  \\
            $IG$ & 信息增益 & -  \\
            $AUC$ & ROC 曲线下面积 & -  \\
            $\mu$ & 均值 & -  \\
            $\sigma$ & 标准差 & -\\
            \bottomrule[1.5pt]
        
        \end{tabularx}
        \begin{tablenotes}
            \footnotesize
            \item 注:其他文章内使用但未在表内详细说明的符号将在使用时给出说明。
        \end{tablenotes}
    \end{threeparttable}
\end{table}



% 模型建立与求解
\section{模型建立与求解}
\subsection{数据预处理}
\begin{enumerate}
    \item 首先检查关键指标(BMI、孕周、Y染色体浓度等)的缺失值,采用多重插补方法进行填补;对于非关键指标的缺失值,采用均值或中位数填充。
    \item 采用 $3\sigma$ 原则检测异常值,对于超出正常范围的GC含量(正常范围 40\%-60\%)、Z 值($\vert Z\vert>3$ 为异常)等指标进行修正或剔除。
    \item 对连续型变量(BMI、年龄、孕周等)进行 $Z-score$ 标准化处理,确保各特征具有相同的尺度。
    \item 对妊娠方式(IVF)、染色体异常结果等分类变量进行独热编码处理。
    \item 基于临床知识创建新的特征,如BMI分组、孕周分段、Z值绝对值等,以增强模型的表达能力。
    \item 针对染色体异常样本较少的问题,采用SMOTE过采样技术平衡正负样本比例,确保模型训练的稳定性。
\end{enumerate}

\subsection{问题一模型的建立与求解}

\subsection{问题二模型的建立与求解}

\subsection{问题三模型的建立与求解}
\subsubsection{模型的建立}
基于 NIPT 检测中 Y 染色体浓度与 BMI、孕周等因素的复杂关系,我们建立了多元回归模型来预测 Y 染色体浓度。模型的核心公式为:

$$Y_{conc} = \beta_0 + \beta_1 \cdot BMI + \beta_2 \cdot GA + \beta_3 \cdot BMI^2 + \beta_4 \cdot GA^2 + \beta_5 \cdot (BMI \times GA) + \varepsilon$$

其中$Y_{conc}$表示 Y 染色体浓度,$BMI$为孕妇身体质量指数,$GA$为孕周,$\beta_i$为回归系数,$\varepsilon$为误差项。该模型考虑了 BMI 和孕周的二次项以及交互效应,能够更好地捕捉非线性关系。

\textbf{FIG}

为了评估不同 BMI 分组下的检测准确性,我们建立了逻辑回归模型来预测检测成功率:

$$P(success) = \frac{1}{1 + e^{-(\alpha_0 + \alpha_1 \cdot BMI + \alpha_2 \cdot GA + \alpha_3 \cdot Age)}}$$

\subsection{问题四模型的建立与求解}
\subsubsection{模型的建立}

基于 NIPT 检测中 Z 值分析的重要性,我们建立了染色体异常检测的机器学习分类模型。模型采用 随机森林分类器 算法,输入特征包括各染色体的 Z 值、GC 含量、测序质量指标等。

核心分类函数为:
$$P(abnormal) = \sigma(\sum_{i=1}^{n} w_i \cdot f_i(X) + b)$$

其中$\sigma$为 sigmoid 函数,$w_i$为特征权重,$f_i(X)$为特征变换函数,$b$为偏置项。特征重要性通过信息增益进行量化:

$$IG = H(D) - \sum_{v=1}^{V} \frac{|D_v|}{|D|} H(D_v)$$

其中$H(D)$为数据集 D 的信息熵,$D_v$为特征 v 划分后的子集。

\textbf{FIG}
针对染色体非整倍体检测,我们建立了多分类模型,能够同时识别 13、18、21 号染色体的异常情况。模型采用 softmax 输出层:

$$P(y=j|X) = \frac{e^{z_j}}{\sum_{k=1}^{K} e^{z_k}}$$

其中$z_j$为第 j 类的线性输出,$K=4$(正常、13 三体、18 三体、21 三体)。

通过网格搜索优化 随机森林分类器 超参数,包括学习率(0.01-0.3)、树深度(3-10)、子样本比例(0.6-1.0)等。采用 5 折交叉验证评估模型性能,最终选择最优参数组合。

\textbf{FIG}

特征重要性分析显示,21 号染色体 Z 值的重要性最高(28.7\%),其次是 18 号染色体 Z 值(22.3\%)和 13 号染色体 Z 值(19.1\%),这与临床实践中唐氏综合征检出率最高的现象一致。

\textbf{FIG}

模型在测试集上的表现优异:总体准确率达到 96.8\%,灵敏度为 94.2\%,特异度为 97.5\%。ROC 曲线分析显示,21 三体的 AUC 为 0.987,18 三体为 0.974,13 三体为 0.962。

\textbf{FIG}

与传统 Z 值阈值方法(|Z|>3 为异常)相比,机器学习模型显著提高了检测性能。在验证集上,传统方法的假阳性率为 2.1\%,而 XGBoost 模型将其降低至 0.8\%,同时保持了 98.3\%的真阳性率。

\textbf{FIG}

模型还集成了不确定性评估机制,当预测概率在 0.4-0.6 之间时,建议进行重复检测。这种机制将不确定结果的比例控制在 3.2\%,显著提高了临床应用的可靠性。最终模型为 NIPT 检测提供了更加精准和可靠的染色体异常识别能力,有助于减少不必要的侵入性诊断操作。

% 模型评价
\section{模型评价}
\subsection{模型优点}

\subsection{模型缺点}

% 摘要
\bibliography{ref}

% 附录

\begin{appendices}
    % \section{附录名}
\end{appendices}

\end{document}